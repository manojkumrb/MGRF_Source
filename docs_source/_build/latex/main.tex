%% Generated by Sphinx.
\def\sphinxdocclass{report}
\documentclass[a4paper,10pt,english]{sphinxmanual}
\ifdefined\pdfpxdimen
   \let\sphinxpxdimen\pdfpxdimen\else\newdimen\sphinxpxdimen
\fi \sphinxpxdimen=.75bp\relax

\PassOptionsToPackage{warn}{textcomp}
\usepackage[utf8]{inputenc}
\ifdefined\DeclareUnicodeCharacter
% support both utf8 and utf8x syntaxes
  \ifdefined\DeclareUnicodeCharacterAsOptional
    \def\sphinxDUC#1{\DeclareUnicodeCharacter{"#1}}
  \else
    \let\sphinxDUC\DeclareUnicodeCharacter
  \fi
  \sphinxDUC{00A0}{\nobreakspace}
  \sphinxDUC{2500}{\sphinxunichar{2500}}
  \sphinxDUC{2502}{\sphinxunichar{2502}}
  \sphinxDUC{2514}{\sphinxunichar{2514}}
  \sphinxDUC{251C}{\sphinxunichar{251C}}
  \sphinxDUC{2572}{\textbackslash}
\fi
\usepackage{cmap}
\usepackage[T1]{fontenc}
\usepackage{amsmath,amssymb,amstext}
\usepackage{babel}



\usepackage{times}
\expandafter\ifx\csname T@LGR\endcsname\relax
\else
% LGR was declared as font encoding
  \substitutefont{LGR}{\rmdefault}{cmr}
  \substitutefont{LGR}{\sfdefault}{cmss}
  \substitutefont{LGR}{\ttdefault}{cmtt}
\fi
\expandafter\ifx\csname T@X2\endcsname\relax
  \expandafter\ifx\csname T@T2A\endcsname\relax
  \else
  % T2A was declared as font encoding
    \substitutefont{T2A}{\rmdefault}{cmr}
    \substitutefont{T2A}{\sfdefault}{cmss}
    \substitutefont{T2A}{\ttdefault}{cmtt}
  \fi
\else
% X2 was declared as font encoding
  \substitutefont{X2}{\rmdefault}{cmr}
  \substitutefont{X2}{\sfdefault}{cmss}
  \substitutefont{X2}{\ttdefault}{cmtt}
\fi


\usepackage[Bjarne]{fncychap}
\usepackage{sphinx}

\fvset{fontsize=\small}
\usepackage{geometry}

% Include hyperref last.
\usepackage{hyperref}
% Fix anchor placement for figures with captions.
\usepackage{hypcap}% it must be loaded after hyperref.
% Set up styles of URL: it should be placed after hyperref.
\urlstyle{same}
\addto\captionsenglish{\renewcommand{\contentsname}{Contents:}}

\usepackage{sphinxmessages}
\setcounter{tocdepth}{2}



\title{mGRF Documentation}
\date{Nov 26, 2019}
\release{V0.1}
\author{Manoj}
\newcommand{\sphinxlogo}{\vbox{}}
\renewcommand{\releasename}{Release}
\makeindex
\begin{document}

\pagestyle{empty}
\sphinxmaketitle
\pagestyle{plain}
\sphinxtableofcontents
\pagestyle{normal}
\phantomsection\label{\detokenize{index::doc}}



\chapter{Alphabetical list of functions and scripts with description}
\label{\detokenize{index:module-mGRF}}\label{\detokenize{index:alphabetical-list-of-functions-and-scripts-with-description}}\index{mGRF (module)@\spxentry{mGRF}\spxextra{module}}\index{contourDomainPlot() (in module mGRF)@\spxentry{contourDomainPlot()}\spxextra{in module mGRF}}

\begin{fulllineitems}
\phantomsection\label{\detokenize{index:mGRF.contourDomainPlot}}\pysiglinewithargsret{\sphinxbfcode{\sphinxupquote{contourDomainPlot}}}{\emph{fem}, \emph{domainID}, \emph{userData}, \emph{colourBar}, \emph{varargin}}{}
A helper fucntion to create contourplot of a given mesh (domain) within
fem structure of VRM for data specified by the user.
\begin{quote}\begin{description}
\item[{Parameters}] \leavevmode\begin{itemize}
\item {} 
\sphinxstyleliteralstrong{\sphinxupquote{fem}} \textendash{} fem structure from VRM after loading the mesh

\item {} 
\sphinxstyleliteralstrong{\sphinxupquote{domainID}} \textendash{} Domain ID of the current mesh file in the fem structure

\item {} 
\sphinxstyleliteralstrong{\sphinxupquote{userdata}} \textendash{} vector of values to visualise, length = nNode X 1

\item {} 
\sphinxstyleliteralstrong{\sphinxupquote{colourBar}} \textendash{} 0 or 1, creates a colourBar when = 1

\item {} 
\sphinxstyleliteralstrong{\sphinxupquote{vargin}} \textendash{} typically ax, axis to plot the figure, if empty a new axis is created

\end{itemize}

\end{description}\end{quote}

Returns:
The coutourplot of the mesh with user given values ar mesh nodes

\end{fulllineitems}

\index{getAssembledKe() (in module mGRF)@\spxentry{getAssembledKe()}\spxextra{in module mGRF}}

\begin{fulllineitems}
\phantomsection\label{\detokenize{index:mGRF.getAssembledKe}}\pysiglinewithargsret{\sphinxbfcode{\sphinxupquote{getAssembledKe}}}{\emph{fem}, \emph{domainID}}{}
A function to assemble the stiffness matrix of a given mesh
element=element strucutre
\begin{quote}\begin{description}
\item[{Parameters}] \leavevmode\begin{itemize}
\item {} 
\sphinxstyleliteralstrong{\sphinxupquote{fem}} \textendash{} fem structure from VRM after loading the mesh

\item {} 
\sphinxstyleliteralstrong{\sphinxupquote{domainID}} \textendash{} Domain ID of the current mesh file in the fem structure

\end{itemize}

\item[{Returns}] \leavevmode
Ke the assemble stiffness matrix

\end{description}\end{quote}

\end{fulllineitems}

\index{getElementStifness() (in module mGRF)@\spxentry{getElementStifness()}\spxextra{in module mGRF}}

\begin{fulllineitems}
\phantomsection\label{\detokenize{index:mGRF.getElementStifness}}\pysiglinewithargsret{\sphinxbfcode{\sphinxupquote{getElementStifness}}}{\emph{fem}, \emph{eleId}}{}
Helper function for {\hyperref[\detokenize{index:mGRF.getAssembledKe}]{\sphinxcrossref{\sphinxcode{\sphinxupquote{getAssembledKe()}}}}}

\end{fulllineitems}

\index{getMgrfDev() (in module mGRF)@\spxentry{getMgrfDev()}\spxextra{in module mGRF}}

\begin{fulllineitems}
\phantomsection\label{\detokenize{index:mGRF.getMgrfDev}}\pysiglinewithargsret{\sphinxbfcode{\sphinxupquote{getMgrfDev}}}{\emph{mGRF}, \emph{meshCoord}, \emph{keyPointsID}, \emph{meshStiffMat}, \emph{nSample}}{}
The main function to simulate non-ideal parts using morphing-Gaussian Random
Fields(mGRF). All options to the generation of non-ideal parts are to be provided in the
mGRF structure. The mGRF structure has to main inputs:
\begin{enumerate}
\sphinxsetlistlabels{\arabic}{enumi}{enumii}{}{.}%
\item {} 
mGRF.HypParmOpt.Type - corresponding to the different ways to input the hyper-parameters of the Gaussian Random Field

\item {} 
mGRF.NIdev.Type - corresponding to the type of non-ideal deviatons to be slimulated

\end{enumerate}
\begin{quote}\begin{description}
\item[{Parameters}] \leavevmode\begin{itemize}
\item {} 
\sphinxstyleliteralstrong{\sphinxupquote{meshCoord}} \textendash{} coordinates of all mesh nodes an nNodes X 3 matrix

\item {} 
\sphinxstyleliteralstrong{\sphinxupquote{keyPointsID}} \textendash{} node ID of all the key points

\item {} 
\sphinxstyleliteralstrong{\sphinxupquote{meshStiffMat}} \textendash{} The stiffness matrix for the whole mesh

\item {} 
\sphinxstyleliteralstrong{\sphinxupquote{nSample}} \textendash{} Number of non-ideal part instances to be simulated

\item {} 
\sphinxstyleliteralstrong{\sphinxupquote{mGRF.HypParmOpt.Type}} \textendash{} ‘measData’ \textbar{} ‘manual’ \textbar{} ‘load’

\item {} 
\sphinxstyleliteralstrong{\sphinxupquote{Options\_specific\_to'measData'}} \textendash{} 

\item {} 
\sphinxstyleliteralstrong{\sphinxupquote{mGRF.HypParmOpt.devPatterns}} \textendash{} nInstances x nNodes matrix of non-ideal  deviations

\item {} 
\sphinxstyleliteralstrong{\sphinxupquote{Options\_specific\_to'manual'}} \textendash{} 

\item {} 
\sphinxstyleliteralstrong{\sphinxupquote{mGRF.HypParmOpt.sn}} \textendash{} noise standard dev

\item {} 
\sphinxstyleliteralstrong{\sphinxupquote{mGRF.HypParmOpt.lScale}} \textendash{} characteristic length scale, 3X1 vector for x,yand z directions

\item {} 
\sphinxstyleliteralstrong{\sphinxupquote{mGRF.HypParmOpt.sf}} \textendash{} scaling factor (set to 1 by default)

\item {} 
\sphinxstyleliteralstrong{\sphinxupquote{Options\_specific\_to'file'}} \textendash{} 

\item {} 
\sphinxstyleliteralstrong{\sphinxupquote{mGRF.HypParmOpt.File}} \textendash{} name of the .mat file containing optimised hyper parameter values for a batch of deviations\%

\item {} 
\sphinxstyleliteralstrong{\sphinxupquote{mGRF.NIdev.Type}} \textendash{} ‘dent’\textbar{}’flange’\textbar{}’bending’\textbar{}’formErr’, String defining the type of non-ideal deformatios

\item {} 
\sphinxstyleliteralstrong{\sphinxupquote{mGRF.NIdev.Probability}} \textendash{} confidence value that max form error is less than specified value (between 0-1)

\item {} 
\sphinxstyleliteralstrong{\sphinxupquote{mGRF.NIdev.MaxFormError}} \textendash{} Maximum specified form error

\item {} 
\sphinxstyleliteralstrong{\sphinxupquote{mGRF.NIdev.NBasis}} \textendash{} Number of basis to use for interpolating the covariance matric for whole mesh

\item {} 
\sphinxstyleliteralstrong{\sphinxupquote{Options\_specific\_to'bending'}} \textendash{} 

\item {} 
\sphinxstyleliteralstrong{\sphinxupquote{mGRF.NIdev.Bending.ID}} \textendash{} Vector of length 2 representing two nodes forming the bending axis

\item {} 
\sphinxstyleliteralstrong{\sphinxupquote{mGRF.NIdev.Bending.Theta}} \textendash{} Bending angleabout the axis in degrees

\item {} 
\sphinxstyleliteralstrong{\sphinxupquote{Options\_specific\_to'dent'\textbar{}'flange'\_local\_deformations}} \textendash{} 

\item {} 
\sphinxstyleliteralstrong{\sphinxupquote{mGRF.NIdev.Local.ID}} \textendash{} vector containing node IDs of all nodes being manipulated

\item {} 
\sphinxstyleliteralstrong{\sphinxupquote{mGRF.NIdev.Local.Dev}} \textendash{} nID X 1, vector of local deformation of key points

\end{itemize}

\item[{Returns}] \leavevmode
dev - The nNodes X nSamples matrix of non-ideal part deviations

\end{description}\end{quote}

\begin{sphinxadmonition}{note}{Note:}
The function depends on \sphinxhref{http://www.gaussianprocess.org/gpml/code/matlab/doc/index.html}{gp Toolbox} and should be in the matlab path before
the {\hyperref[\detokenize{index:mGRF.getMgrfDev}]{\sphinxcrossref{\sphinxcode{\sphinxupquote{getMgrfDev()}}}}} is called.
\end{sphinxadmonition}

\end{fulllineitems}



\begin{fulllineitems}
\phantomsection\label{\detokenize{index:mGRF.mGRF_main}}\pysigline{\sphinxbfcode{\sphinxupquote{mGRF\_main}}}
A script demonstrating the abilites of non-ideal part modelling and
simulation using the mGRF methodology
It shows the modelling and simulation of non-ideal parts of an automotive
door inner for:
\begin{enumerate}
\sphinxsetlistlabels{\arabic}{enumi}{enumii}{}{.}%
\item {} 
Local deformation of flange.

\item {} 
Global deformation of bending.

\end{enumerate}

It also demonstrated various options to find the optimum covariance
function parameters (hyper-parameters), namely:
\begin{enumerate}
\sphinxsetlistlabels{\arabic}{enumi}{enumii}{}{.}%
\item {} 
Learn from cloud of point data

\item {} 
Load parameters from a file.

\item {} 
Set the parameters manually.

\item {} 
Use a default set of parameters.

\end{enumerate}

\end{fulllineitems}

\index{setCovStruct() (in module mGRF)@\spxentry{setCovStruct()}\spxextra{in module mGRF}}

\begin{fulllineitems}
\phantomsection\label{\detokenize{index:mGRF.setCovStruct}}\pysiglinewithargsret{\sphinxbfcode{\sphinxupquote{setCovStruct}}}{\emph{type}, \emph{varargin}}{}
A function to help set options of the mean and covariance fuction
according to gp toolbox. Matern covariance function is used as defalult
throughout
\begin{quote}\begin{description}
\item[{Parameters}] \leavevmode\begin{itemize}
\item {} 
\sphinxstyleliteralstrong{\sphinxupquote{type}} \textendash{} ‘default’\textbar{}’user’ string

\item {} 
\sphinxstyleliteralstrong{\sphinxupquote{If\_type\_is'user'\_three\_vectors\_are\_expected}} \textendash{} 

\item {} 
\sphinxstyleliteralstrong{\sphinxupquote{sn}} \textendash{} 1x1 scalar, noise standard dev (makes surface points pass through key points if zero). Default is 0.001

\item {} 
\sphinxstyleliteralstrong{\sphinxupquote{lscale}} \textendash{} 3x1 characteristic length scale in x, y and z respectively, default is 20, 20, 10 mm

\item {} 
\sphinxstyleliteralstrong{\sphinxupquote{sf}} \textendash{} 1x1 scalar, latent function standard dev or output scaling factor, Default is 1

\end{itemize}

\item[{Returns}] \leavevmode
Covariance function parameters in syntax compatible with gp toolbox

\end{description}\end{quote}

\end{fulllineitems}




\end{document}